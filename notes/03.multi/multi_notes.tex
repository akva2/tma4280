\documentclass[11pt]{article}
\bibliographystyle{plain}
\usepackage{geometry} % see geometry.pdf on how to lay out the page. There's lots.
\usepackage{amsmath,amssymb} 
\usepackage{epsfig,epsf,subfigure}
\geometry{a4paper} 


%\documentclass{article}

\begin{document}
 
\LARGE
\begin{center}
TMA4280: Introduction to Supercomputing
\end{center}
\vspace{1in}

\begin{center}
{\bf Multiprocessor systems}
\end{center}

\Large
\vspace{0.5in}
\begin{center}
January 2011 \\
Revised January 2014
\end{center}

\vspace{0.5in}

\begin{center}
\copyright Einar M. R{\o}nquist \\
Department of Mathematical Sciences\\
NTNU, N-7491 Trondheim, Norway\\
All rights reserved
\end{center}

\large

\newpage

\section{Supercomputing}

Supercomputing represents the high performance segment of the overall computing market. 
This segment has traditionally been driven by grand challenge problems 
in science and engineering. This is still the situation, although new areas and 
new applications are constantly being added to the list of problems 
where supercomputing is necessary (or at least highly desirable). 

A strong motivation for using supercomputing is the possibility of performing
larger and more realistic simulations, e.g., by solving more detailed mathematical 
models in science and engineering. Hence, the design of the largest computing 
systems have traditionally be driven by challenges in science and engineering. 

About 30-40 years ago, supercomputers were typically specially designed 
vector processors capable of performing operations on vector data. 
A single instruction could operate on multiple data elements (vectors) at once,
and the hardware comprised custom-made chips. Because of the low production 
volume and the costly development effort, each supercomputer was very expensive. 

Supercomputers in the 80's and 90's also used fast, expensive, custom-made chips, 
and combined a few of these in a multiprocess context. 
However, starting in the late 80's, microprocessor-based supercomputers 
became more and more popular. These systems were also called 
massively parallel processors (MPPs) because they used many more processors
than traditional vector supercomputers. The advantage of the 
microprocessor-based supercomputers was the use of standard, 
off-the-shelf  microprocessors
instead of using costly, custom-made chips. 
The supercomputer market today is dominated by MPP systems. 
A supercomputer system typically combines thousands of processors. 
Some of the largest systems comprise hundreds of thousands of processors. 

Supercomputing is very resource demanding in terms of
floating point operations, memory and storage requirement, 
as well as visualization capabilities. 
Thus, some of the important challenges related to the development and use of a 
multiprocessor system concern: 
\begin{enumerate}
\item communication between the processors\\
(network topology, memory access, programming models); 
\item development of suitable computational methods or algorithms\\
(e.g., domain decomposition algorithms);
\item scalability (both in terms of hardware and algorithms); 
\item handling of large volumes of data (storage and visualization). 
\end{enumerate}

\newpage
We add some additional remarks regarding scalability. 
Let $T_P$ be the time to solve a given problem on $P$ processors,
where time refers to wall clock time.  
We define the speedup, $S_P$, as 
\begin{align}
S_P = \frac{T_1}{T_P},
\label{multi:speedup}
\end{align}
i.e., as the ratio between the solution time on a single processor 
divided by the solution time on $P$ processors. 
Ideally, we would like $S_P$ to be equal to $P$, implying the 
$P$ processors should be able to solve the problem $P$ times as fast 
as a single processor. A more realistic situation is depicted in Figure 
\ref{fig:scalability}: for small systems (i.e., when $P$ is small), we typically 
get good speedup for a fixed problem. As more processors are added, 
the computational task per processor is reduced, while the communication 
overhead between the processors typically increases. Hence, after a certain 
number of processors, it does not pay off to add more processors. 

Note that the above situation gives a too pessimistic view of supercomputing. 
The assumption about a fixed problem size is typically not correct.  
With the availability of larger computing systems, the problem size 
is typically also increased. Many problems solved on large systems
are of a size that cannot be solved in a single processor context, either because
of the storage requirement, because of the computational cost, or both. 
From this view point, the definition of speedup in (\ref{multi:speedup}) 
must be used with some care. 

\begin{figure}[htbp]
  \begin{center}
    \includegraphics[scale=0.6]{scalability}
  \end{center}
\caption{Ideal speedup ($S_P = P$) and realistic speedup. 
}
\label{fig:scalability}
\end{figure}

Finally, a couple of important reminders: when working in a multiprocessor
environment, the single-processor performance is still of utmost importance. 
In particular, we recall some of the possibilities of parallelism within a single processor: 
bit-level, instruction-level, pipelining, and superscalar capabilities 
(or multiple floating point units). 

\section{Multiprocessor systems: organization}

According to Almasi and Gottlieb (1989), a parallel computer is a collection 
of processing elements which communicate and cooperate to solve a problem fast. 
This definition raises some immediate questions: 
\begin{itemize}
\item How many processing elements should we use?
\item How should the processing elements communicate?
\item Will the parallel computer be scalable?
\item How powerful should a processor be?
\item What about programming?
\end{itemize}

Figure \ref{fig:multiprocess1} and \ref{fig:multiprocess2} show two 
examples of organizations. In Figure \ref{fig:multiprocess1} each processor
has a local cache and can access memory modules via some type of interconnect. 
In this case, all the memory modules can be accessed directly by all the processors. 
This organization is referred to as global or shared memory access. 
In Figure \ref{fig:multiprocess2} each processor has a local memory 
and can only access information in other memory modules via 
an interconnecting network. This organization is referred to as 
distributed memory access. 
Each organization has its strengths and weaknesses. 
We will return to some of these issues later. 

\begin{figure}[htbp]
  \begin{center}
    \includegraphics[scale=0.7]{multiprocess1}
  \end{center}
\caption{A parallel computer with global memory access. 
}
\label{fig:multiprocess1}
\end{figure}

\begin{figure}[htbp]
  \begin{center}
    \includegraphics[scale=0.7]{multiprocess2}
  \end{center}
\caption{A parallel computer with distributed memory access. 
}
\label{fig:multiprocess2}
\end{figure}

\subsection{Uniform memory access}

There exist several ways to achieve global memory access. 
One type of systems is referred to as SMP: Symmetric Multi-Processor. 
This is a configuration where all the memory locations are "equidistant"
in the sense that the memory access time is the same. 

An example of an SMP is a bus-based organization; 
see Figure \ref{fig:bus_organization}. 
This system has a broadcast interconnect (similar to Ethernet),
it is inexpensive, and it is easy to add processors. 
The disadvantage with a bus organization is the very limited scalability, 
which is due to the fact that the aggregate bandwidth is fixed. 
The use of caches could potentially alleviate some of this problem, 
however, then the question of how to achieve cache coherency 
(or consistency) arises. 

\begin{figure}[htbp]
  \begin{center}
    \includegraphics[scale=0.7]{bus_organization}
  \end{center}
\caption{A bus-based organization. The global memory is accessible
to each processor.
}
\label{fig:bus_organization}
\end{figure}

Another example of an SMP is a crossbar or a switch-based organization. 
Similar to a bus-based system, cache coherency is needed (e.g., via broadcast). 
A crossbar has a better scalability than a bus organization. 
The aggregate bandwidth is increased, however, adding a processor
becomes more and more expensive as the system size grows
(because the number of parts increases).

\begin{figure}[htbp]
  \begin{center}
    \includegraphics[scale=0.6]{crossbar}
  \end{center}
\caption{A crossbar interconnect. The global memory is accessible to each processor. 
}
\label{fig:crossbar}
\end{figure}

In summary, there is a scalability problem with the interconnect both 
with a bus organization and with a crossbar. In the case of the bus, the 
scalability issue is due to the fixed aggregate bandwidth; 
in the case of the crossbar, the scalability relates to the cost. 
An example of a compromise between these two issues is the multistage interconnect; 
see Figure \ref{fig:multistage_interconnect}.

\begin{figure}[htbp]
  \begin{center}
    \includegraphics[scale=0.6]{multistage_interconnect}
  \end{center}
\caption{A multi-stage interconnect. The global memory is accessible to each processor. 
}
\label{fig:multistage_interconnect}
\end{figure}

\subsection{Non-uniform memory access}

An alternative to an SMP-organization is a NUMA-organization 
(Non-Uniform Memory Access); see Figure \ref{fig:NUMA}. 
The last supercomputers at NTNU, the Cray T3E, the SGI Origin, 
and the current IBM supercomputer, all represent examples  of this type of organization.

\begin{figure}[htbp]
  \begin{center}
    \includegraphics[scale=0.6]{NUMA}
  \end{center}
\caption{A NUMA organization.  
}
\label{fig:NUMA}
\end{figure}

The Cray T3E had caches which were only used to hold data and 
instructions from local memory. The computer had no mechanism 
to keep the caches consistent with the global address space. 
The computer was an example of a non-coherent shared memory machine. 
In principle, any processor could read/write from/to any memory location
in the global address space. In practice, however, messages were used to 
transfer data from one local memory module to another. 
The programming model was thus based on message passing. 
We will return to this model later. 

On the SGI Origin, data from any memory location could be replicated 
into any of the caches. Hardware support existed to keep the caches consistent. 
This is also referred to as a ccNUMA organization 
(cache coherent Non-Uniform Memory Access). 
Any processor could read/write from/to any memory location. 
This feature could be exploited in a shared memory programming model. 
However, note that a message passing programming model could still be used
on the SGI. 
We will return to a discussion of the current supercomputer at NTNU later. 

\subsection{Distributed memory access}

An example of a distributed memory organization is the mesh interconnect 
as depicted in Figure \ref{fig:2Dmesh}. Each processor can access 
its own local memory similar to the single processor case. 
However, data from the local memory associated 
with the neighboring processors are obtained by message passing. 
The two first bits of the message are used 
to determine in which direction (North, South, East, West) to move. 
The two bits are then stripped off, and the message proceeds to the next 
step on the two-dimensional mesh. The message itself is trailing behind 
while the connection is being established. This approach is referred to as 
{\em wormhole routing} and was developed by William Dally at MIT. 
The mesh interconnect was used on the Intel Paragon and the Intel Delta 
supercomputers (in the 90's).  

\begin{figure}[htbp]
  \begin{center}
    \includegraphics[scale=0.6]{2Dmesh}
  \end{center}
\caption{A 2D mesh interconnect. Each circle represents a processing element: 
a CPU, local cache, and a local memory, i.e., a single processor. 
Such a processing element is also referred to as a node.  
}
\label{fig:2Dmesh}
\end{figure}
 
An improved version of the mesh interconnect is the 2D toroid; 
see Figure \ref{fig:2Dtoroid}. Compared to the 2D mesh, there are
wrap-around connections in each spatial direction (horizontally and vertically)
in order to reduce the worst-case hop count. 

Further extensions of the mesh interconnect are the 3D mesh or 3D toroid 
network topology (e.g., used on the Cray T3D and Cray T3E supercomputers). 

\begin{figure}[htbp]
  \begin{center}
    \includegraphics[scale=0.55]{2Dtoroid}
  \end{center}
\caption{A 2D toroid interconnect. 
}
\label{fig:2Dtoroid}
\end{figure}

Finally, we mention the hypercube organization; see Figure \ref{fig:hypercube}. 
This type of interconnect was attractive before the wormhole-routing. 
The whole message was typically passed through one hop before the next hop 
could start, something which resulted in very long communication times. 

\begin{figure}[htbp]
  \begin{center}
    \includegraphics[scale=0.45]{hypercube}
  \end{center}
\caption{A $d$-dimensional hypercube has $2^d$ nodes (or processing elements),
and the longest "distance" (or hops) between any two nodes is $d$. 
}
\label{fig:hypercube}
\end{figure}

\subsection{MIMD - supercomputers}

In the following, we will focus on MIMD architectures. 
MIMD is an acronym for Multiple Instructions Multiple Data, and 
refers to the fact that there are multiple instruction streams 
(one per processor or node) and multiple data streams (one per processor or node). 
A MIMD supercomputer is a general multiprocessor. 

Alternative architectures are SISD - Single Instruction Single Data 
(a classical workstation), and SIMD - Single Instruction Multiple  Data 
(sometimes called an array processor or a vector processor). 
An example of a SIMD interface is HPF - High Performance FORTRAN.

There are different types of MIMD computers: 
\begin{enumerate}
\item MIMD with {\em shared} uniform memory  (SMP, UMA), or \\
MIMD with {\em shared} nonuniform memory  (NUMA, ccNUMA); 
\item MIMD with distributed memory (and message passing). 
\end{enumerate}

Examples of the first category are depicted in Figure \ref{fig:SMP}
and Figure \ref{fig:ccNUMA}. In both cases, the global 
 address space is available to every processor; 
hence, the term shared memory. 
A programming model for shared memory machines exists,
and the current de facto standard is OpenMP; see 
\textrm{http://www.openmp.org}.

\begin{figure}[htbp]
  \begin{center}
    \includegraphics[scale=0.5]{SMP}
  \end{center}
\caption{A Symmetric Multi-Processor (SMP) --- sometimes also referred to as a node. 
A node typically comprises 16-64 processors with a cache-coherent uniform memory. 
}
\label{fig:SMP}
\end{figure}

\begin{figure}[htbp]
  \begin{center}
    \includegraphics[scale=0.7]{ccNUMA}
  \end{center}
\caption{A ccNUMA supercomputer may comprise several SMP nodes, 
all linked together by an interconnecting network. All the caches are kept 
coherent, however, the memory access is only uniform within a single SMP; 
the memory access time varies between different SMPs.
}
\label{fig:ccNUMA}
\end{figure}

For the second class of MIMD computers, only local address space is 
directly accessible to each processor; see Figure \ref{fig:MIMD}. 
Access to non-local memory is achieved via message passing. 
The standard communication library used for message passing is MPI
(Message Passing Interface). 

We remark that programs written for distributed memory machines 
often run very well on shared memory machines. 
This is the reason why we will focus on a distributed memory programming model 
in this course, even though we may be running our programs 
on a shared memory machine. We thus make a distinction between 
the {\emph{programming model}} we choose and the particular {\emph {machine}} we use. 
On the other hand, if we had chosen a shared memory programming model, 
we could only use the program on a shared memory machine. 

\begin{figure}[htbp]
  \begin{center}
    \includegraphics[scale=0.7]{MIMD}
  \end{center}
\caption{A distributed memory computer. Only the local address space is 
accessible to each processor. Different processors can only communicate via 
explicit message passing. 
}
\label{fig:MIMD}
\end{figure}

%\clearpage
\section{The current supercomputer at NTNU}

We now give a brief overview of the current supercomputer at NTNU, 
a SGI (Altix) supercomputer based on the Intel i7 Sandy bridge microprocessor.

\subsection{The Sandy Bridge microprocessor}

Multi-core systems is a recent trend in chip design. 
The i7 microprocessor represents an example of an octa-core chip. 
This means that a single chip integrates 8 processors (or cores) 
into a single integrated circuit; see Figure \ref{fig:p5}.
A single chip therefore represents 8 physical processors in the 
sense discussed earlier. Each core is hyperthreaded meaning they
can handle two instruction streams simultaneously. Thus, each chip has
8 \emph{physical} cores but 16 \emph{logical} cores. The idea behind
this design is to attempt to hide memory latencies in the system
by having one instruction stream do calculations, while the other
stream is waiting for data, but there are no extra resources for calculation
available. If your code is already efficiently supplying data to one
of the instruction streams, the use of hyperthreading might actually harm
performance.

\begin{figure}[htbp]
  \begin{center}
    \includegraphics[scale=0.5]{sandy}
  \end{center}
\caption{The figure depicts a quad-core Sandy Bridge microprocessor (or chip).
         The author was unable to locate an image of an octa-core chip.
}
\label{fig:p5}
\end{figure}

\subsection{Interconnecting chips to form larger SMPs}

Multiple Sandy Bridge chips can be connected to form larger modules in SMP configurations.
On \emph{vilje}, two chips are run in SMP mode on each compute node. Thus, up to
16 (32) processors share memory.

\subsection{Interconnecting nodes to form a distributed SMP}

Multiple nodes may also be connected to form clusters or distributed SMPs. 
In this case, data from one node to another must be passed across 
a high bandwidth low-latency switch network. 
A system based on multiple nodes must therefore be treated as a 
distributed memory system (at least globally). 

\subsection{Key data for \texttt{vilje}}

The current supercomputer at NTNU, \texttt{vilje}, is based on 1404 nodes. 
 The total number of processors (or cores) is therefore 22464. 
 
Each node represents a shared memory system with 2 octa-core Sandy Bridge chips
which share 32 GB memory (although a few nodes have 128 GB memory). 

Each i7 processor operates at a clock rate of 2.6 GHz. 
The size of the L1 cache (3 clocks) is 32 kbyte for data and 32 kbyte for instructions. 
The size of the L2 cache (8 clocks) is 256 kbyte, while the size of the shared off-chip L3 cache
is 20 Mbyte. 

\section{Programming models}

We have seen that a node (with 16 physical processors) on \texttt{vilje} 
represents a shared memory system where the aggregate memory for the node 
is globally available to all the 16 processors. A shared memory programming model (i.e., OpenMP) 
can therefore be used within a single node.

If we want to develop programs which can run on more than one node, we 
need to use message-passing (i.e., MPI). This is also the programming model we will 
emphasize in this course. Even though each node represents a shared memory 
system, the message-passing programming model may still be used within a node.
However, the opposite is not true: a shared programming model cannot be used on a "pure"
distributed memory system (e.g., on a PC cluster). 

Note that a system like \texttt{vilje}, which represents a shared memory 
system within a node, and a distributed memory system across the nodes, 
can also be programmed using both programming models within a single program. 

\clearpage
\section{Message-passing}

Message-passing is fundamentally processor-to-processor communication. 
Only a local, unique memory is directly available to each processor. 
Both local and remote processes must cooperate in order to exchange 
data and/or synchronize (at least originally --- some changes have been made 
in the extended version MPI-2). 

Note that message-passing is a good way to use distributed shared memory
machines (ccNUMA) because it provides a way to express memory/data locality. 

Some of the key advantages of the message-passing model are: 
\begin{itemize}
\item {\em portability}: the model can be used on a collection of 
homogeneous or heterogeneous processors connected by a fast or a slow 
communication network; 
\item {\em performance}: the approach exploits data locality, as well as the 
availability of a large, aggregate memory; 
\item {\em expressiveness}: a limited communcation library suffices for most applications. 
\end{itemize}

The Message Passing Interface (MPI) is the de facto standard for message passing. 
It is a {\em library}, not a language. MPI provides efficiency, portability and functionality. 
It represents a standardized communcation library running on a vast number of 
machines and architectures. Figure \ref{fig:message_passing_model} illustrates the message passing model. 

The original (1994) MPI-library represents the message passing model 
where both the local and remote processes cooperate, 
e.g., via a send and receive operation. 
MPI-2 represents an extension of MPI where features like 
one-sided messages, parallel I/O, etc. are included. 




\begin{figure}[htbp]
  \begin{center}
    \includegraphics[scale=0.6]{message_passing_model}
  \end{center}
\caption{The message passing model. A number of processes, $P_0, P_1, \ldots, 
P_{n-1}$, are coupled
together via a fast or a slow communication network. 
Each process has a local and unique memory/cache, and each 
process is associated with a particular computational task. 
The individual processes must communicate via explicit message passing.
A message consists of an "envelope" which contains sufficient information 
about whether and when to open the message, as well as information 
regarding how to interpret the "body" of the message (the actual data). 
Note that the message is the only means of exchanging data between the processes 
and/or syncronizing the processes. 
}
\label{fig:message_passing_model}
\end{figure}


The MPI operations can be classified in a few types of operations: 
\begin{itemize}
\item one-to-one; 
\item one-to-all; 
\item all-to-one; 
\item all-to-all. 
\end{itemize}
The first type is also referred to as point-to-point operations
(send and receive), while the last three types are collective operations. 

When we here talk about "all", we generally mean all processes
$P_0, P_1, \ldots, P_{n-1}$ within a group of $n$ processes. 
Such a group defines a {\em communicator}, and the particular 
process number is referred to as the {\em rank} within that communicator. 
The default is to let all processes be members of the same (default) communicator. 
However, it is also possible to have some of the processes be members
of one communicator (or group), while others be members of a different communicator.
In this context, "all" means all the processes {\em within}�
a particular communicator. 

Finally, the collective operations can be further broken down into the 
following categories: 
\begin{itemize}
\item data movement (broadcast; gather/scatter); 
\item collective computation (max/min; sum; etc.).
\end{itemize}

We will later explain in more detail the various MPI operations. 
A good way to learn MPI is by implementing a few simple examples. 
The whole library contains about 125 functions. However, as few as 6 may 
suffice for some problems. You only need to learn the functions needed for 
your particular problem. You may not have  to learn the details of the whole 
library even for advanced applications. 

\newpage

\subsection{An example}

We now discuss a brief example of a program where the MPI library is used. 
The program listed below does the following: processor 0 sends a text message
"Hello, world" to all the other processors. The other processors receive
the message and all processors print out the message together with 
the their own process number. 

Below is a listing of a program to do this. We show the C version below and 
the  corresponding FORTRAN version on the next page. 

\begin{verbatim}



#include <stdio.h>
#include "mpi.h"

int main(int argc, char **argv)
{
      int rank, size, tag, i; 
      MPI_Status status; 
      char message[20];

      MPI_Init (&argc, &argv);
      MPI_Comm_size (MPI_COMM_WORLD, &size);
      MPI_Comm_rank (MPI_COMM_WORLD, &rank);

      tag = 100; 
         
      if (rank == 0) {
             strcpy (message, "Hello, world");
             for (i=1; i < size; i++) {
                 MPI_Send (message, 13, MPI_CHAR, i, tag, MPI_COMM_WORLD);
             }
      }
      else {
          MPI_Recv (message, 13, MPI_CHAR, 0, tag, MPI_COMM_WORLD, &status);
      }
         
      printf ( "node %d :  %13s\n", rank, message);
         
      MPI_Finalize();
      
      return(0);
}
\end{verbatim}


\newpage

Below is a listing of the  FORTRAN version. 
\begin{verbatim}



 program hello
 include 'mpif.h'

 integer rank, size, ierror, tag, status(MPI_STATUS_SIZE)
 character(12) message

 call MPI_INIT (ierror);
 call MPI_COMM_SIZE (MPI_COMM_WORLD, size, ierror);
 call MPI_COMM_RANK (MPI_COMM_WORLD, rank, ierror);

 tag = 100; 
         
 if (rank .eq. 0) then
     message = 'Hello, world'
     do i=1, size-1
        call MPI_SEND (message, 12, MPI_CHARACTER, i, tag, 
$                                  MPI_COMM_WORLD, ierror)
     enddo
 else
     call MPI_RECV (message, 12, MPI_CHARACTER, 0, tag, 
$                          MPI_COMM_WORLD, status, ierror)
 endif
         
 print*, 'node', rank, ':' message
         
 call MPI_Finalize (ierror)
 end

\end{verbatim}


\newpage

We run the C version of the program on the SGI Origin (the previous supercomputer at NTNU) using $P=4$ and $P=8$ processors.
The output from the program in these two cases are: \\ \\

\underline{$\mathbf{P=4}$}
\begin{verbatim}

  node 0 :   Hello, world
  node 1 :   Hello, world
  node 3 :   Hello, world
  node 2 :   Hello, world






\end{verbatim}

\underline{$\mathbf{P=8}$}
\begin{verbatim}

  node 0 :   Hello, world
  node 5 :   Hello, world
  node 1 :   Hello, world
  node 7 :   Hello, world
  node 2 :   Hello, world
  node 3 :   Hello, world
  node 6 :   Hello, world
  node 4 :   Hello, world

\end{verbatim}

\newpage

Let us now comment on some of the statements here. We start with
\begin{verbatim}
   #include <stdio.h>
   #include "mpi.h"
\end{verbatim}
The first statement is just a standard statement about including the 
header file associated with string (character) operations and I/O. 
The second statement is required if we want to use the MPI library 
in our program. In C, we need the statement
\begin{verbatim}
   #include "mpi.h"
\end{verbatim}
The equivalent statement in FORTRAN is
\begin{verbatim}
   include 'mpi.f'
\end{verbatim}
The MPI header file provides basic MPI definitions and MPI data types. 

The first MPI statement needs to be:
\begin{verbatim}
      MPI_Init (&argc, &argv);
\end{verbatim}
This statement should only be called once. 
The last MPI statement is always
\begin{verbatim}
      MPI_Finalize();
\end{verbatim}
This statement does not have to be the very last statement in your program,
but it needs to be the last MPI statement. The statement ensures a clean exit. 

Note that the command line arguments are passed to the C version of 
\textrm{MPI\_Init}. The corresponding FORTRAN version reads: 
\begin{verbatim}
      call MPI_INIT (ierror);
\end{verbatim}
Similar to a standard subroutine call in FORTRAN, a call to an MPI operation 
also starts with \textrm{"call"}. The parameter \textrm{"ierror"} is an integer 
and will return an error code in case something goes wrong. 
The C version also returns an error code. Instead of the statement used 
in the program listed above, we could alternatively have written
\begin{verbatim}
      errorcode = MPI_Init (&argc, &argv);
\end{verbatim}
In this case, the parameter  \textrm{errorcode} (an integer) 
will contain an error code if something goes wrong. 

In general, any MPI statement in C has the format
\begin{verbatim}
      errorcode = MPI_Xxxxx (parameters);
\end{verbatim}
or, simply
\begin{verbatim}
      MPI_Xxxxx (parameters);
\end{verbatim}
The particular MPI operation is given by "Xxxxx" where the name 
of the operation always starts with a capital letter and the remaining 
letters are lower-case. The number and type of parameters vary from 
operation to operation. For example, MPI\_Finalize does not have 
any parameters at all. 

In FORTRAN, any MPI statement has the format.
\begin{verbatim}
      call MPI_XXXXX (parameters, ierror)
\end{verbatim}
where the parameter "ierror" returns an error code. Note that the 
name of the particular MPI operation is always in capital letters. 

The next two statements after the initialization of MPI are: 
\begin{verbatim}
      MPI_Comm_size (MPI_COMM_WORLD, &size);
      MPI_Comm_rank (MPI_COMM_WORLD, &rank);
\end{verbatim}
The first of these statements returns the total number of MPI processes, 
while the second one returns the individual process number (the "rank"). \\
MPI\_COMM\_WORLD is the default name of the communicator 
(the "universe") and which include all the processes. 
This can be changed in order to create several separate "universes." 

Note that the program listed in this example runs separately and independently 
on every processor on a multiprocessor. We also refer to this  as 
SPMD - Single Program Multiple Data. All the problems we will study in this 
course will be of this type: the program running on each processor 
will be the same for all the processors. However, the data each processor will 
operate on will typically be different. The synchronization of the program will be implicit
via the MPI operations. We will return to this issue later.
 
When this program runs on a particular processor, the program does not 
automatically know how many other processors are involved; 
this issue is taken care of by the MPI operation MPI\_Comm\_size. 
In the multiprocess context, the program running on an individual processor
does not automatically know what its associated process number is
(who am I?); this issue is taken care of by the MPI operation MPI\_Comm\_rank. 

Let us now proceed to the statements 
\begin{verbatim}
      if (rank == 0) {
             strcpy (message, "Hello, world");
\end{verbatim}
The if-statement will only be true for one of the processes, namely, 
the process with process number 0. This is also referred to as the "root" process. 
On the root process, the string "Hello, world" is copied into the variable "message". 

For all the other processes, the if-statement will not be true, and the program 
execution will move on to the MPI statement MPI\_Recv. 
This means that all of the other processes will be waiting for process 0 
to send them data. Process 0 sends data to the other processes in the 
loop
\begin{verbatim}
       for (i=1; i < size; i++) {
           MPI_Send (message, 13, MPI_CHAR, i, tag, MPI_COMM_WORLD);
       }
\end{verbatim}
Several comments are in order here. First, note that process 0 sends out
the same message (string) to all the other processes. This is done in a loop. 
However, note that this loop corresponds to a {\em sequential} execution meaning 
that a message will be sent to process 1 before process 2 etc. 

Let us now discuss the particular format in the parameter list for the 
operation MPI\_Send. The general format for this operation is
\begin{verbatim}
    MPI_Send (start, count, datatype, dest, tag, comm);
\end{verbatim}
The first three parameters (start, count, datatype) represent {\em the data}, 
while the last three parameters (dest, tag, comm) represent {\em the envelope}; 
see Figure \ref{fig:message_passing_model}.

We now explain all these parameters in some more detail:

\begin{verbatim}
start     : initial address of the send buffer 
count     : number of elements sent (of type "datatype")
datatype  : e.g., MPI_INT, MPI_FLOAT, MPI_CHAR etc. 
dest      : destination process (integer)
tag       : integer message identifier (e.g., MPI_ANY_TAG)
comm      : an ordered group of communication processes (same for send & recv)  
\end{verbatim}

Only process 0 sends a message. All the other processes are waiting to receive
a message. This is expressed by the MPI operation MPI\_Recv. 
The general format for this operation is 
\begin{verbatim}
    MPI_Recv (start, count, datatype, source, tag, comm, &status);
\end{verbatim}
The first three parameters (start, count, datatype) represent {\em the data}, 
while the last three parameters (dest, tag, comm, \&status) represent {\em the envelope}; 
again, see Figure \ref{fig:message_passing_model}.

Most of the parameters are similar to the MPI\_Send operation: 
\begin{verbatim}
start     : initial address of the receive buffer 
count     : number of elements sent (of type "datatype")
datatype  : e.g., MPI_INT, MPI_FLOAT, MPI_CHAR etc. 
source    : source process (integer), i.e., the process sending the message
tag       : integer message identifier (e.g., MPI_ANY_TAG)
comm      : an ordered group of communication processes (same for send & recv) 
&status   : a structure providing information on the completed communication 
\end{verbatim}

Note that in the small example program, an explicit "source" in MPI\_Recv is given, 
namely, process 0. Alternatively, we could have used
MPI\_ANY\_SOURCE since each process in receive mode only expects one message. 
Similarly, we note that the "tag" is explicitly given. Alternatively, we could  have used 
MPI\_ANY\_TAG since this parameter is not critical in our case. 

One additional comment regarding the send and receive statements. 
This is an example of point-to-point communication. There exists several 
versions of send and receive. The type used here is called {\em blocking} 
send and receive. 
This means that the program running on processor 0 cannot proceed 
until each message has been safely sent and the send buffer can be 
safely used again. Similarly, the program running on each of the other 
processors cannot proceed until the expected message has been received
in the receive buffer. 

Let us now comment on the parameter "count" used in the send and receive statements. 
In the C version, the count parameter is set equal to 13, while it is 12 in the FORTRAN version. 
The reason for this is that the strcpy function in C will append "$\backslash$0" (the null character) at the end 
of the message. This is a symbol which indicates the end of the string.
 Even though "Hello, world" comprises 12 letters (one byte per letter), 
the message buffer in C requires an extra byte of memory. 

The parameter "datatype" in the send and receive operations has to be the same. 
In this case, the type is MPI\_CHAR (in FORTRAN the corresponding data type is 
called MPI\_CHARACTER). 
This is one of several predefined data types. Other important ones include
MPI\_DOUBLE and MPI\_INT. 

Finally, note that the output from the programs is not in a sequential order. 
The order may also change if we run the program over again. 

\end{document}

The next section will discuss the particular details related to the 
MPI standard and operations. 


\begin{figure}[htbp]
  \begin{center}
    \includegraphics[scale=1]{ieee}
  \end{center}
  \caption{Each floating point has a binary representation with 
three fields: $S$ denotes the sign of the number, 
$E$ is an exponent, and $F$ is the fraction part of the mantissa.
}
\label{fig:fp}
\end{figure}


\begin{center}
{\bf Table 1} \\
Number of bits for common floating point precisions
\end{center}
\begin{center}
\begin{tabular}{|c|c|c|c|c|c|c|c|} \hline
Precision & $S$ & $E$ & $F$ & Total  
  \\ \hline
Single & $1$ & $8$ & $23$ & 32  
  \\ \hline
Double & $1$ & $11$ & $52$ & 64 
  \\ \hline
\end{tabular}
\end{center}

\vspace{.2in}
\noindent {\bf Exercise 4}. Propose a way to avoid the above limitation. 





