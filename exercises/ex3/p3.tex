\documentclass[11pt]{article}
\bibliographystyle{plain}
\usepackage{geometry}
\usepackage{amsmath,amssymb} 
\usepackage{epsfig,epsf,subfigure}
\geometry{a4paper} 

\begin{document}
\centerline{\LARGE\textbf{TMA4280 Introduction to Supercomputing}}

\vspace*{5ex}

\centerline{\Large\textbf{Problem Set 3}}

\large

\vspace*{8ex}
\centerline{Einar M. R{\o}nquist}

\vspace*{4ex}
\centerline{Spring 2011}

\subsection*{Exercise 1}

Check out the the web site http://www.top500.org/  describing the 
top 500 supercomputers in the world. In particular, 
read more details about the top 10 supercomputers in the world
(what type of computer, technical
specifications, etc.). What is meant by the LINPACK
benchmark performance?

\subsection*{Exercise 2}

\begin{enumerate}
\item What limits the scalability of a bus-based interconnect?
\item How are the individual processors connected using a crossbar?
\item How are the individual processors connected using a mesh?
\item What is the difference between a shared-memory and a distributed 
memory architecture?
\item Is the memory access time uniform for an SMP?
\item What is the difference between a NUMA and a ccNUMA architecture?
\end{enumerate}

\subsection*{Exercise 3}

How many bytes are sent in each of the three messages listed below \\(here given in C, but this is not important)?
\begin{verbatim}
MPI_Send(buffer1, 80, MPI_CHAR, dest, tag, MPI_COMM_WORLD);

MPI_Send(buffer2, 1024, MPI_INT, dest, tag, MPI_COMM_WORLD);

MPI_Send(buffer3, 1024, MPI_DOUBLE, dest, tag, MPI_COMM_WORLD);

\end{verbatim}

\subsection*{Exercise 4}
Is it true that a unique tag must be specified each time MPI\_Recv is called?

\subsection*{Coding task}
Implement the program from exercise 2 using MPI.
The program should run on any number of processes.
Hint; You should partition the matrix in column strips. 

\end{document}
